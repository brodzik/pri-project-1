\documentclass[a4paper, 12pt]{article}
\usepackage[utf8]{inputenc}
\usepackage[T1]{fontenc}
\usepackage{titlesec}
\usepackage{listings}
\usepackage{tikz}
\usetikzlibrary{shapes, arrows, positioning, fit, backgrounds}

\tikzstyle{rect} = [draw, rectangle, rounded corners, fill=white, minimum width=1.5cm, minimum height=0.7cm]
\tikzstyle{fillonly} = [draw, rectangle, minimum width=7cm, minimum height=3cm, fill=black!10]
\tikzstyle{line} = [draw, -latex]

\lstset{basicstyle=\ttfamily, keepspaces=true}

\titleformat*{\section}{\Large\bfseries}

\title{\bf [PRI] Projekt pierwszy}
\author{Adrian Brodzik}

\begin{document}

\maketitle

\section*{Zadanie}
Napisz program, który wypisze linię znaków wejściowych zastępując każdy ciąg \textit{białych} znaków pojedynczym znakiem ukośnika, zaś ewentualne każde wystąpienie ukośnika w linii zastąpi podwójnym ukośnikiem.
\\
\\
Dla danych wejściowych:\\
\lstinline{bfvrhowev h7u893  njio nvfrowe/vtgrw         vvv}
\\
\\
Poprawnym wynikiem jest:\\
\lstinline{bfvrhowev/h7u893/njio/nvfrowe//vtgrw/vvv}

\section*{Problem}
Identyfikować ciągi białych znaków i zastępować je dokładnie jednym ukośnikiem; znajdować istniejące już wcześniej ukośniki i je duplikować.

\section*{Rozwiązanie}
Iterujemy po każdym znaku $c_{i}$ z linii znaków wejściowych. Na wyjście wypisujemy znak po znaku. Wypisujemy pojedynczy ukośnik, gdy $c_{i}$ jest znakiem białym i $c_{i-1}$ nie jest znakiem białym. Wypisujemy dwa ukosniki, gdy $c_{i}$ jest ukośnikiem. Dla znaków niebędącymi znakami białymi ani ukośnikami, wypisujemy $c_{i}$.

\section*{Biblioteka standardowa}
Użyto funkcję \texttt{printf} ze standardowej biblioteki \texttt{stdio.h} do wyświetlania komunikatów dla użytkownika oraz do wypisywania wyjścia programu.

\section*{Schemat działania}
\begin{center}
\begin{tikzpicture}
\node [rect] (start) {start};
\node [rect, below=0.5cm of start] (input) {wejście linii znaków};
\node [fillonly, below=0.5cm of input] (fill) {};
\node [below=0.6cm of input] (iterate) {iteracja po każdym znaku};
\node [rect, below=0.2cm of iterate] (replace) {zastąpienie znaku (jeżeli konieczne)};
\node [rect, below=0.33cm of replace] (print) {wypisanie znaku na wyjście};
\node [rect, below=0.5cm of fill] (stop) {stop};

\path [line] (start) -- (input);
\path [line] (input) -- (fill);
\path [line] (replace) -- (print);
\path [line] (fill) -- (stop);
\end{tikzpicture}
\end{center}

\section*{Testowanie}
Należy zwrócić uwagę na odpowiednie formatowanie wejścia programu, np. uruchomienie
\\\\
\lstinline{> ./app /some string  with   spaces    /}\\
\lstinline{Usage: app [string]}\\
\lstinline{Example: ./app "some string"}
\\\\
skutkuje błędem. Spacje w tym przypadku rozdzielają kolejne argumenty, tzn. nie wprowadzono jednej linii znaków, ale cztery. Poprawne uruchomienie
\\\\
\lstinline{> ./app "/some string  with   spaces    /"}\\
\lstinline{//some/string/with/spaces///}

\section*{Podsumowanie}
Projekt prosty służący jako wprowadzenie do języka programowania C; zapoznanie się ze składnią, typami i funkcjami sterującymi.

\end{document}
